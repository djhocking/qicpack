\nonstopmode{}
\documentclass[a4paper]{book}
\usepackage[times,inconsolata,hyper]{Rd}
\usepackage{makeidx}
\usepackage[utf8,latin1]{inputenc}
% \usepackage{graphicx} % @USE GRAPHICX@
\makeindex{}
\begin{document}
\chapter*{}
\begin{center}
{\textbf{\huge Package `QICpack'}}
\par\bigskip{\large \today}
\end{center}
\begin{description}
\raggedright{}
\item[Type]\AsIs{Package}
\item[Title]\AsIs{Model selection for generalized estimating equations using QIC}
\item[Version]\AsIs{0.9}
\item[Date]\AsIs{2012-12-10}
\item[Author]\AsIs{Daniel J. Hocking}
\item[Maintainer]\AsIs{Daniel J. Hocking }\email{dhocking@unh.edu}\AsIs{}
\item[Description]\AsIs{This package contains the function 'qic', which calculates Pan's (2001) quasi likelihood information criteria (QIC) for models generated from the gee pack R package. The package also contains a function 'qictab' that compares a list of geeglm models from geepack and calculates delta QIC, QIC model weights, and cumulative model weights.}
\item[License]\AsIs{GPL (>= 3)}
\item[URL]\AsIs{}\url{http://github.com/djhocking/qicpack}\AsIs{}
\item[Depends]\AsIs{MASS, geepack, stats}
\end{description}
\Rdcontents{\R{} topics documented:}
\inputencoding{utf8}
\HeaderA{QICpack-package}{Model selection for generalized estimating equations using QIC}{QICpack.Rdash.package}
\aliasA{QICpack}{QICpack-package}{QICpack}
\keyword{package}{QICpack-package}
%
\begin{Description}\relax
Package contains functions to calculate QIC (Pan 2001) and compare generalized estimating equation models based on QIC.
\end{Description}
%
\begin{Details}\relax

\Tabular{ll}{
Package: & QICpack\\{}
Type: & Package\\{}
Version: & 0.9\\{}
Date: & 2012-12-10\\{}
License: & GPL (>= 3)\\{}
}
qic(model.R) calculates QIC for a single model fit with geeglm in geepack

qictab(cand.set, modnames) calculates QIC values for a list of models and compares the models based on QIC.
\end{Details}
%
\begin{Author}\relax
Daniel J. Hocking

Maintainer: Daniel J. Hocking <dhocking@unh.edu>
\end{Author}
%
\begin{References}\relax
Burnham, K. P. and D. R. Anderson. 2002. Model selection and multimodel inference: a practical information-theoretic approach. Second edition. Springer Science and Business Media, Inc., New York.
Pan, W. 2001. Akaike's information criterion in generalized estimating equations. Biometrics 57:120-125.
\end{References}
\inputencoding{utf8}
\HeaderA{dietox}{Toxicity data from geepack}{dietox}
\keyword{datasets}{dietox}
%
\begin{Description}\relax
Heavy metal toxicity data from geepack
\end{Description}
%
\begin{Usage}
\begin{verbatim}
data(dietox)
\end{verbatim}
\end{Usage}
%
\begin{Format}
A data frame with 861 observations on the following 7 variables.
\begin{description}

\item[\code{Weight}] a numeric vector
\item[\code{Feed}] a numeric vector
\item[\code{Time}] a numeric vector
\item[\code{Pig}] a numeric vector
\item[\code{Evit}] a numeric vector
\item[\code{Cu}] a numeric vector
\item[\code{Litter}] a numeric vector

\end{description}

\end{Format}
%
\begin{References}\relax
Hojsgaard, S., U. Halekoh, and J. Yan. 2006. The R package geepack for generalized estimating equations. Journal of Statistical Software 15:1-11.
\end{References}
\inputencoding{utf8}
\HeaderA{qic}{Calculates QIC (Pan 2001) for model generated using geeglm in geepack}{qic}
\keyword{\textbackslash{}textasciitilde{}QIC}{qic}
\keyword{\textbackslash{}textasciitilde{}GEE}{qic}
%
\begin{Description}\relax
This function calculates the quasilikelihood information criteria (QIC; Pan 2001) for model generated using geeglm in geepack. The QIC is intended as an equivalent of AIC for generalized estimating equations (GEE-PA).
\end{Description}
%
\begin{Usage}
\begin{verbatim}
qic(model.R)
\end{verbatim}
\end{Usage}
%
\begin{Arguments}
\begin{ldescription}
\item[\code{model.R}] 
model.R is the fitted gee model from geeglm within geepack

\end{ldescription}
\end{Arguments}
%
\begin{Value}





Returns QIC, log quasilikelihood, trace (eqivalent to K in AIC), px in a data frame
\end{Value}
%
\begin{Author}\relax
Daniel J. Hocking
\end{Author}
%
\begin{References}\relax
Pan, W. 2001. Akaike's information criterion in generalized estimating equations. Biometrics 57:120-125.
\end{References}
%
\begin{Examples}
\begin{ExampleCode}
##---- Should be DIRECTLY executable !! ----
##-- ==>  Define data, use random,
##--	or do  help(data=index)  for the standard data sets.

\end{ExampleCode}
\end{Examples}
\inputencoding{utf8}
\HeaderA{qictab}{Calculates QIC (Pan 2001) for a list of fitted objects from geeglm within geepack and outputs a table relative model fits}{qictab}
\keyword{\textbackslash{}textasciitilde{}GEE}{qictab}
\keyword{\textbackslash{}textasciitilde{}QIC}{qictab}
%
\begin{Description}\relax
Calculates QIC, delta QIC, QIC weights similiar to the aictab function within the AICcmodavg package. The function outputs a table of of the relative fit of each model in desending order.
\end{Description}
%
\begin{Usage}
\begin{verbatim}
qictab(cand.set, modnames, sort = TRUE)
\end{verbatim}
\end{Usage}
%
\begin{Arguments}
\begin{ldescription}
\item[\code{cand.set}] 
A list of models (candidate set) fit using geeglm from geepack

\item[\code{modnames}] 
Names of the models in the candidate set list

\item[\code{sort}] 
If TRUE sorts the output table in desending order from best model to least supported

\end{ldescription}
\end{Arguments}
%
\begin{Value}





Function returns a data table of model name, QIC, Log Quaislikelihood, Trace, px, delta QIC, QIC model weights, and cummulative model weight
\end{Value}
%
\begin{Author}\relax
Daniel J. Hocking <dhocking@unh.edu>
\end{Author}
%
\begin{References}\relax
Pan, W. 2001. Akaike's information criterion in generalized estimating equations. Biometrics 57:120-125.

Burnham, K. P. and D. R. Anderson. 2002. Model selection and multimodel inference: a practical information-theoretic approach. Second edition. Springer Science and Business Media, Inc., New York.

\end{References}
%
\begin{Examples}
\begin{ExampleCode}
##---- Should be DIRECTLY executable !! ----
##-- ==>  Define data, use random,
##--	or do  help(data=index)  for the standard data sets.


\end{ExampleCode}
\end{Examples}
\printindex{}
\end{document}
